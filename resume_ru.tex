% vim: set textwidth=120:

% Example CV based on the 1.5-column-cv template. Main features:
% * uses the Roboto font family and IcoMoon icon set;
% * doesn't use colours, different font weights are used instead for styling;
% * because the CV fits on one page, header and footer is empty, since there isn't much useful info to put there;
% * includes a photo.
\documentclass[a4paper,10pt]{article}


% package imports
% ---------------

\usepackage[british]{babel} % for correct language and hyphenation and stuff
\usepackage{calc}           % for easier length calculations (infix notation)
\usepackage{enumitem}       % for configuring list environments
\usepackage{fancyhdr}       % for setting header and footer
\usepackage{fontspec}       % for fonts
\usepackage{geometry}       % for setting margins (\newgeometry)
\usepackage{graphicx}       % for pictures
\usepackage{microtype}      % for microtypography stuff
\usepackage{xcolor}         % for colours
\usepackage{hyperref}


% margin and column widths
% ------------------------

% margins
\newgeometry{left=10mm,right=10mm,top=10mm,bottom=10mm}

% width of the gap between left and right column
\newlength{\cvcolumngapwidth}
\setlength{\cvcolumngapwidth}{3.5mm}

% left column width
\newlength{\cvleftcolumnwidth}
\setlength{\cvleftcolumnwidth}{44.5mm}

% right column width
\newlength{\cvrightcolumnwidth}
\setlength{\cvrightcolumnwidth}{\textwidth-\cvleftcolumnwidth-\cvcolumngapwidth}

% set paragraph indentation to 0, because it screws up the whole layout otherwise
\setlength{\parindent}{0mm}


% style definitions
% -----------------
% style categories explanation:
% * \cvnameXXX is used for the name;
% * \cvsectionXXX is used for section names (left column, accompanied by a horizontal rule);
% * \cvtitleXXX is used for job/education titles (right column);
% * \cvdurationXXX is used for job/education durations (left column);
% * \cvheadingXXX is used for headings (left column);
% * \cvmainXXX (and \setmainfont) is used for main text;
% * \cvruleXXX is used for the horizontal rules denoting sections.

% font families
\defaultfontfeatures{Ligatures=TeX} % reportedly a good idea, see https://tex.stackexchange.com/a/37251

\newfontfamily{\cvnamefont}{Roboto Medium}
\newfontfamily{\cvsectionfont}{Roboto Medium}
\newfontfamily{\cvtitlefont}{Roboto Regular}
\newfontfamily{\cvdurationfont}{Roboto Light Italic}
\newfontfamily{\cvheadingfont}{Roboto Regular}
\setmainfont{Roboto Light}

% colours
\definecolor{cvnamecolor}{HTML}{000000}
\definecolor{cvsectioncolor}{HTML}{000000}
\definecolor{cvtitlecolor}{HTML}{000000}
\definecolor{cvdurationcolor}{HTML}{000000}
\definecolor{cvheadingcolor}{HTML}{000000}
\definecolor{cvmaincolor}{HTML}{000000}
\definecolor{cvrulecolor}{HTML}{000000}

\color{cvmaincolor}

% styles
\newcommand{\cvnamestyle}[1]{{\Large\cvnamefont\textcolor{cvnamecolor}{#1}}}
\newcommand{\cvsectionstyle}[1]{{\normalsize\cvsectionfont\textcolor{cvsectioncolor}{#1}}}
\newcommand{\cvtitlestyle}[1]{{\large\cvtitlefont\textcolor{cvtitlecolor}{#1}}}
\newcommand{\cvdurationstyle}[1]{{\small\cvdurationfont\textcolor{cvdurationcolor}{#1}}}
\newcommand{\cvheadingstyle}[1]{{\normalsize\cvheadingfont\textcolor{cvheadingcolor}{#1}}}


% inter-item spacing
% ------------------

% vertical space after personal info and standard CV items
\newlength{\cvafteritemskipamount}
\setlength{\cvafteritemskipamount}{5mm plus 1.25mm minus 1.25mm}

% vertical space after sections
\newlength{\cvaftersectionskipamount}
\setlength{\cvaftersectionskipamount}{2mm plus 0.5mm minus 0.5mm}

% extra vertical space to be used when a section starts with an item with a heading (e.g. in the skills section),
% so that the heading does not follow the section name too closely
\newlength{\cvbetweensectionandheadingextraskipamount}
\setlength{\cvbetweensectionandheadingextraskipamount}{1mm plus 0.25mm minus 0.25mm}


% intra-item spacing
% ------------------

% vertical space after name
\newlength{\cvafternameskipamount}
\setlength{\cvafternameskipamount}{3mm plus 0.75mm minus 0.75mm}

% vertical space after personal info lines
\newlength{\cvafterpersonalinfolineskipamount}
\setlength{\cvafterpersonalinfolineskipamount}{2mm plus 0.5mm minus 0.5mm}

% vertical space after titles
\newlength{\cvaftertitleskipamount}
\setlength{\cvaftertitleskipamount}{1mm plus 0.25mm minus 0.25mm}

% value to be used as parskip in right column of CV items and itemsep in lists (same for both, for consistency)
\newlength{\cvparskip}
\setlength{\cvparskip}{0.5mm plus 0.125mm minus 0.125mm}

% set global list configuration (use parskip as itemsep, and no separation otherwise)
\setlist{parsep=0mm,topsep=0mm,partopsep=0mm,itemsep=\cvparskip}


% CV commands
% -----------

% creates a "personal info" CV item with the given left and right column contents, with appropriate vertical space after
% @param #1 left column content (should be the CV photo)
% @param #2 right column content (should be the name and personal info)
\newcommand{\cvpersonalinfo}[2]{
    % left and right column
    \begin{minipage}[t]{\cvleftcolumnwidth}
        \vspace{0mm} % XXX hack to align to top, see https://tex.stackexchange.com/a/11632
        \raggedleft #1
    \end{minipage}% XXX necessary comment to avoid unwanted space
    \hspace{\cvcolumngapwidth}% XXX necessary comment to avoid unwanted space
    \begin{minipage}[t]{\cvrightcolumnwidth}
        \vspace{0mm} % XXX hack to align to top, see https://tex.stackexchange.com/a/11632
        #2
    \end{minipage}

    % space after
    \vspace{\cvafteritemskipamount}
}

% typesets a name, with appropriate vertical space after
% @param #1 name text
\newcommand{\cvname}[1]{
    % name
    \cvnamestyle{#1}

    % space after
    \vspace{\cvafternameskipamount}
}

% typesets a line of personal info beginning with an icon, with appropriate vertical space after
% @param #1 parameters for the \includegraphics command used to include the icon
% @param #2 icon filename
% @param #3 line text
\newcommand{\cvpersonalinfolinewithicon}[3]{
    % icon, vertically aligned with text (see https://tex.stackexchange.com/a/129463)
    \raisebox{.5\fontcharht\font`E-.5\height}{\includegraphics[#1]{#2}}
    % text
    #3

    % space after
    \vspace{\cvafterpersonalinfolineskipamount}
}

% creates a "section" CV item with the given left column content, a horizontal rule in the right column, and with
% appropriate vertical space after
% @param #1 left column content (should be the section name)
\newcommand{\cvsection}[1]{
    % left and right column
    \begin{minipage}[t]{\cvleftcolumnwidth}
        \raggedleft\cvsectionstyle{#1}
    \end{minipage}% XXX necessary comment to avoid unwanted space
    \hspace{\cvcolumngapwidth}% XXX necessary comment to avoid unwanted space
    \begin{minipage}[t]{\cvrightcolumnwidth}
        \textcolor{cvrulecolor}{\rule{\cvrightcolumnwidth}{0.3mm}}
    \end{minipage}

    % space after
    \vspace{\cvaftersectionskipamount}
}

% creates a standard, multi-purpose CV item with the given left and right column contents, parskip set to cvparskip
% in the right column, and with appropriate vertical space after
% @param #1 left column content
% @param #2 right column content
\newcommand{\cvitem}[2]{
    % left and right column
    \begin{minipage}[t]{\cvleftcolumnwidth}
        \raggedleft #1
    \end{minipage}% XXX necessary comment to avoid unwanted space
    \hspace{\cvcolumngapwidth}% XXX necessary comment to avoid unwanted space
    \begin{minipage}[t]{\cvrightcolumnwidth}
        \setlength{\parskip}{\cvparskip} #2
    \end{minipage}

    % space after
    \vspace{\cvafteritemskipamount}
}

% typesets a title, with appropriate vertical space after
% @param #1 title text
\newcommand{\cvtitle}[1]{
    % title
    \cvtitlestyle{#1}

    % space after
    \vspace{\cvaftertitleskipamount}
    % XXX need to subtract cvparskip here, because it is automatically inserted after the title "paragraph"
    \vspace{-\cvparskip}
}


% header and footer
% -----------------

% set empty header and footer
\pagestyle{empty}



% preamble end/document start
% ===========================

\begin{document}


% personal info
% -------------

\cvpersonalinfo{
    % photo
    \includegraphics[height=36mm]{img/dan.jpg}
}{
    % name
    \cvname{Даниил Моисеев}
    \cvdurationstyle {Go, C++, Python разработчик}

    % address
    \cvpersonalinfolinewithicon{height=4mm}{img/072-location.pdf}{Москва, Россия}

    % phone number
    \cvpersonalinfolinewithicon{height=4mm}{img/067-phone.pdf}{
        \href{tel:+79156207966}{+7 (915) 620-79-66}
    }

    % telegram
    \cvpersonalinfolinewithicon{height=4mm}{img/telegram.png}{
        \href{https://t.me/ksusonic}{@ksusonic}
    }

    % email address
    \cvpersonalinfolinewithicon{height=4mm}{img/070-envelop.pdf}{
        \href{mailto:ksusonic@ya.ru}{ksusonic@ya.ru}
    }

    % github account
    \cvpersonalinfolinewithicon{height=4mm}{img/github.png}{
        \href{https://github.com/ksusonic}{GitHub}
    }

    % linkedin account
    \cvpersonalinfolinewithicon{height=4mm}{img/linkedin.png}{
        \href{https://www.linkedin.com/in/ksusonic}{Linkedin}
    }
}


\cvsection{О себе}
\cvitem{
    \cvdurationstyle{}
}{
    Backend-энтузиаст с опытом в Golang, C++ и Python.
    Создаю масштабируемые и производительные сервисы, а также проектирую архитектурные решения для эффективного масштабирования приложений.
}

% work experience
% ---------------

\cvsection{Опыт работы}

% Yandex middle
\cvitem{
    \cvdurationstyle{Сентябрь 2021 - Наст. время}
}{
    \cvtitle{Разработчик бэкенда}

    Яндекс, группа разработки видео сценариев Алисы

    \begin{itemize}[leftmargin=*]
        \item Занимаюсь развитием видеосмотрения в умных устройствах (ТВ, Станции), голосовым управлением устройствами. (С++, Kotlin)
    \end{itemize}
}

% Yandex internship
\cvitem{
    \cvdurationstyle{Июнь 2021 -- Сентябрь 2021}
}{
    \cvtitle{Стажер-разработчик}

    Яндекс, поисковый портал

    \begin{itemize}[leftmargin=*]
        \item Занимался разработкой в команде стабильности сервисов: улучшал инструменты для работы с инцидентами внутри Яндекса. (Python)
    \end{itemize}
}

% Kodland
\cvitem{
    \cvdurationstyle{Ноябрь 2020 -- Май 2021}
}{
    \cvtitle{Преподаватель Python}

    Онлайн-школа программирования Kodland

    \begin{itemize}[leftmargin=*]
        \item Обучал детей школьного возраста программированию на Python.
    \end{itemize}
}


% education
% ---------

\cvsection{Образование}

% master's
\cvitem{
    \cvdurationstyle{2023 -- 2025}
}{
    \cvtitle{Высшая школа экономики}

    Машинное обучение и высоконагруженные системы, магистратура
}

% bachelor's
\cvitem{
    \cvdurationstyle{2019 -- 2023}
}{
    Информатика и вычилительная техника, бакалавриат
}

% bachelor's
\cvitem{
    \cvdurationstyle{2021 -- 2022}
}{
    Интеллектуальный анализ данных (майнор)
}

% skills
% ------

\cvsection{Навыки}

\vspace{\cvbetweensectionandheadingextraskipamount}

\cvitem{
    \cvheadingstyle{Golang}
}{
    Заинтересовавший однажды язык стал моим любимым. Прошел курс продвинутой разработки в Яндекс Практикуме, пишу pet-проекты в свободное время.
    \begin{itemize}
        \item gin-gonic, chi, stdlib
        \item SQL, Gorm
        \item gRPC
    \end{itemize}
}

\cvitem{
    \cvheadingstyle{С++}
}{
    В данное время - наиболее используемый язык, используемый в production-проектах.
    \begin{itemize}
        \item C++20, STL
        \item gRPC
    \end{itemize}
}

\cvitem{
    \cvheadingstyle{Python 3}
}{
    Навыки как бэкенд-разработки, так и машинного обучения.
    \begin{itemize}
        \item asyncio, aiohttp, aiogram
        \item Django, FastApi, Flask
        \item Pydantic, SQLAlchemy, alembic, asyncpg
        \item numpy, pandas, sklearn, pytorch
        \item pytest
    \end{itemize}
}

% completely fake skills
\cvitem{
    \cvheadingstyle{Computer Science}
}{
    Linux, Unix
    \begin{itemize}
        \item Protobuf, gRPC
        \item Linux, git
        \item Sql, noSQL
        \item Kubernetes
        \item Docker, CI/CD
        \item RESTful API
        \item AWS
        \item GraphQL
        \item RabbitMQ, Kafka
        \item Prometheus, Grafana
    \end{itemize}
}


% additional info
% ---------------

\cvsection{}

\vspace{\cvbetweensectionandheadingextraskipamount}

% interests
\cvitem{
    \cvheadingstyle{Мои компетенции}
}{
    \begin{itemize}[leftmargin=0.5cm, label={}]
        \item {\textbf{Командная работа:} Я привержен командной работе и умею успешно сотрудничать с коллегами, внося свой вклад в общий успех проекта. Моя способность работать в команде помогает создавать гармоничную рабочую атмосферу и достигать поставленных целей.}
        \item {\textbf{Планирование:} Я умею эффективно планировать свою работу, определять приоритеты и распределять время, чтобы достигать поставленных целей в срок. Моя систематичность и организованность помогают мне быть продуктивным и эффективным в своей деятельности.}
        \item {\textbf{Ориентация на пользователя:} Я всегда помню, что конечный пользователь - главный критик моей работы. Я стремлюсь создавать продукты и решения, которые удовлетворяют потребности и ожидания пользователей, обеспечивая им удобство и удовлетворение.}
        \item {\textbf{Саморазвитие:} Я постоянно стремлюсь к самосовершенствованию и обучению. Я активно изучаю новые технологии, следую за трендами в своей области и посещаю профессиональные курсы и конференции, чтобы быть в курсе последних тенденций и лучших практик в своей области. Мое стремление к постоянному развитию помогает мне оставаться востребованным и успешным профессионалом.}
    \end{itemize}
}

Это резюме сделано с помощью LaTeX и GitHub Actions: github.com/ksusonic/CV

\end{document}
