% vim: set textwidth=120:

% Example CV based on the 1.5-column-cv template. Main features:
% * uses the Roboto font family and IcoMoon icon set;
% * doesn't use colours, different font weights are used instead for styling;
% * because the CV fits on one page, header and footer is empty, since there isn't much useful info to put there;
% * includes a photo.
\documentclass[a4paper,10pt]{article}


% package imports
% ---------------

\usepackage[british]{babel} % for correct language and hyphenation and stuff
\usepackage{calc}           % for easier length calculations (infix notation)
\usepackage{enumitem}       % for configuring list environments
\usepackage{fancyhdr}       % for setting header and footer
\usepackage{fontspec}       % for fonts
\usepackage{geometry}       % for setting margins (\newgeometry)
\usepackage{graphicx}       % for pictures
\usepackage{microtype}      % for microtypography stuff
\usepackage{xcolor}         % for colours
\usepackage{hyperref}


% margin and column widths
% ------------------------

% margins
\newgeometry{left=10mm,right=10mm,top=10mm,bottom=10mm}

% width of the gap between left and right column
\newlength{\cvcolumngapwidth}
\setlength{\cvcolumngapwidth}{3.5mm}

% left column width
\newlength{\cvleftcolumnwidth}
\setlength{\cvleftcolumnwidth}{44.5mm}

% right column width
\newlength{\cvrightcolumnwidth}
\setlength{\cvrightcolumnwidth}{\textwidth-\cvleftcolumnwidth-\cvcolumngapwidth}

% set paragraph indentation to 0, because it screws up the whole layout otherwise
\setlength{\parindent}{0mm}


% style definitions
% -----------------
% style categories explanation:
% * \cvnameXXX is used for the name;
% * \cvsectionXXX is used for section names (left column, accompanied by a horizontal rule);
% * \cvtitleXXX is used for job/education titles (right column);
% * \cvdurationXXX is used for job/education durations (left column);
% * \cvheadingXXX is used for headings (left column);
% * \cvmainXXX (and \setmainfont) is used for main text;
% * \cvruleXXX is used for the horizontal rules denoting sections.

% font families
\defaultfontfeatures{Ligatures=TeX} % reportedly a good idea, see https://tex.stackexchange.com/a/37251

\newfontfamily{\cvnamefont}{Roboto Medium}
\newfontfamily{\cvsectionfont}{Roboto Medium}
\newfontfamily{\cvtitlefont}{Roboto Regular}
\newfontfamily{\cvdurationfont}{Roboto Light Italic}
\newfontfamily{\cvheadingfont}{Roboto Regular}
\setmainfont{Roboto Light}

% colours
\definecolor{cvnamecolor}{HTML}{000000}
\definecolor{cvsectioncolor}{HTML}{000000}
\definecolor{cvtitlecolor}{HTML}{000000}
\definecolor{cvdurationcolor}{HTML}{000000}
\definecolor{cvheadingcolor}{HTML}{000000}
\definecolor{cvmaincolor}{HTML}{000000}
\definecolor{cvrulecolor}{HTML}{000000}

\color{cvmaincolor}

% styles
\newcommand{\cvnamestyle}[1]{{\Large\cvnamefont\textcolor{cvnamecolor}{#1}}}
\newcommand{\cvsectionstyle}[1]{{\normalsize\cvsectionfont\textcolor{cvsectioncolor}{#1}}}
\newcommand{\cvtitlestyle}[1]{{\large\cvtitlefont\textcolor{cvtitlecolor}{#1}}}
\newcommand{\cvdurationstyle}[1]{{\small\cvdurationfont\textcolor{cvdurationcolor}{#1}}}
\newcommand{\cvheadingstyle}[1]{{\normalsize\cvheadingfont\textcolor{cvheadingcolor}{#1}}}


% inter-item spacing
% ------------------

% vertical space after personal info and standard CV items
\newlength{\cvafteritemskipamount}
\setlength{\cvafteritemskipamount}{5mm plus 1.25mm minus 1.25mm}

% vertical space after sections
\newlength{\cvaftersectionskipamount}
\setlength{\cvaftersectionskipamount}{2mm plus 0.5mm minus 0.5mm}

% extra vertical space to be used when a section starts with an item with a heading (e.g. in the skills section),
% so that the heading does not follow the section name too closely
\newlength{\cvbetweensectionandheadingextraskipamount}
\setlength{\cvbetweensectionandheadingextraskipamount}{1mm plus 0.25mm minus 0.25mm}


% intra-item spacing
% ------------------

% vertical space after name
\newlength{\cvafternameskipamount}
\setlength{\cvafternameskipamount}{3mm plus 0.75mm minus 0.75mm}

% vertical space after personal info lines
\newlength{\cvafterpersonalinfolineskipamount}
\setlength{\cvafterpersonalinfolineskipamount}{2mm plus 0.5mm minus 0.5mm}

% vertical space after titles
\newlength{\cvaftertitleskipamount}
\setlength{\cvaftertitleskipamount}{1mm plus 0.25mm minus 0.25mm}

% value to be used as parskip in right column of CV items and itemsep in lists (same for both, for consistency)
\newlength{\cvparskip}
\setlength{\cvparskip}{0.5mm plus 0.125mm minus 0.125mm}

% set global list configuration (use parskip as itemsep, and no separation otherwise)
\setlist{parsep=0mm,topsep=0mm,partopsep=0mm,itemsep=\cvparskip}


% CV commands
% -----------

% creates a "personal info" CV item with the given left and right column contents, with appropriate vertical space after
% @param #1 left column content (should be the CV photo)
% @param #2 right column content (should be the name and personal info)
\newcommand{\cvpersonalinfo}[2]{
    % left and right column
    \begin{minipage}[t]{\cvleftcolumnwidth}
        \vspace{0mm} % XXX hack to align to top, see https://tex.stackexchange.com/a/11632
        \raggedleft #1
    \end{minipage}% XXX necessary comment to avoid unwanted space
    \hspace{\cvcolumngapwidth}% XXX necessary comment to avoid unwanted space
    \begin{minipage}[t]{\cvrightcolumnwidth}
        \vspace{0mm} % XXX hack to align to top, see https://tex.stackexchange.com/a/11632
        #2
    \end{minipage}

    % space after
    \vspace{\cvafteritemskipamount}
}

% typesets a name, with appropriate vertical space after
% @param #1 name text
\newcommand{\cvname}[1]{
    % name
    \cvnamestyle{#1}

    % space after
    \vspace{\cvafternameskipamount}
}

% typesets a line of personal info beginning with an icon, with appropriate vertical space after
% @param #1 parameters for the \includegraphics command used to include the icon
% @param #2 icon filename
% @param #3 line text
\newcommand{\cvpersonalinfolinewithicon}[3]{
    % icon, vertically aligned with text (see https://tex.stackexchange.com/a/129463)
    \raisebox{.5\fontcharht\font`E-.5\height}{\includegraphics[#1]{#2}}
    % text
    #3

    % space after
    \vspace{\cvafterpersonalinfolineskipamount}
}

% creates a "section" CV item with the given left column content, a horizontal rule in the right column, and with
% appropriate vertical space after
% @param #1 left column content (should be the section name)
\newcommand{\cvsection}[1]{
    % left and right column
    \begin{minipage}[t]{\cvleftcolumnwidth}
        \raggedleft\cvsectionstyle{#1}
    \end{minipage}% XXX necessary comment to avoid unwanted space
    \hspace{\cvcolumngapwidth}% XXX necessary comment to avoid unwanted space
    \begin{minipage}[t]{\cvrightcolumnwidth}
        \textcolor{cvrulecolor}{\rule{\cvrightcolumnwidth}{0.3mm}}
    \end{minipage}

    % space after
    \vspace{\cvaftersectionskipamount}
}

% creates a standard, multi-purpose CV item with the given left and right column contents, parskip set to cvparskip
% in the right column, and with appropriate vertical space after
% @param #1 left column content
% @param #2 right column content
\newcommand{\cvitem}[2]{
    % left and right column
    \begin{minipage}[t]{\cvleftcolumnwidth}
        \raggedleft #1
    \end{minipage}% XXX necessary comment to avoid unwanted space
    \hspace{\cvcolumngapwidth}% XXX necessary comment to avoid unwanted space
    \begin{minipage}[t]{\cvrightcolumnwidth}
        \setlength{\parskip}{\cvparskip} #2
    \end{minipage}

    % space after
    \vspace{\cvafteritemskipamount}
}

% typesets a title, with appropriate vertical space after
% @param #1 title text
\newcommand{\cvtitle}[1]{
    % title
    \cvtitlestyle{#1}

    % space after
    \vspace{\cvaftertitleskipamount}
    % XXX need to subtract cvparskip here, because it is automatically inserted after the title "paragraph"
    \vspace{-\cvparskip}
}


% header and footer
% -----------------

% set empty header and footer
\pagestyle{empty}



% preamble end/document start
% ===========================

\begin{document}


% personal info
% -------------

\cvpersonalinfo{
    % photo
    \includegraphics[height=36mm]{dan.jpg}
}{
    % name
    \cvname{Daniil Moiseev}
    \cvdurationstyle {Python, C++ developer}

    % address
    \cvpersonalinfolinewithicon{height=4mm}{img/072-location.pdf}{Russia, Moscow}

    % phone number
    \cvpersonalinfolinewithicon{height=4mm}{img/067-phone.pdf}{
        \href{tel:+79156207966}{+7 (915) 620-79-66}
    }

    % telegram
    \cvpersonalinfolinewithicon{height=4mm}{img/telegram.png}{
        \href{https://t.me/ksusonic}{@ksusonic}
    }

    % email address
    \cvpersonalinfolinewithicon{height=4mm}{img/070-envelop.pdf}{
        \href{mailto:dan@dannyy.ru}{dan@dannyy.ru}
    }

    % github account
    \cvpersonalinfolinewithicon{height=4mm}{img/github-square-512.png}{
        \href{https://github.com/ksusonic}{GitHub}
    }
}


\cvsection{Personal information}
\cvitem{
    \cvdurationstyle{}
}{
    Now I'm developing the Alice - voice assistant by Yandex. I find it challenging, so it stimulates to figure out the better solutions.
    I prefer product development rather than infrastructure. My C++ code is under NDA and pet-projects are on the GitHub.
}

% work experience
% ---------------

\cvsection{Work experience}

% Yandex
\cvitem{
    \cvdurationstyle{September 2021 -- present}
}{
    \cvtitle{Junior backend developer}

    Yandex (Russia), SmartTV backend team

    \begin{itemize}[leftmargin=*]
        \item I'm develioping the video scenario of Alice in smart devices. (С++)
    \end{itemize}
}

% Yandex internship
\cvitem{
    \cvdurationstyle{June 2021 -- September 2021}
}{
    \cvtitle{Intern}

    Yandex, search departament

    \begin{itemize}[leftmargin=*]
        \item Development at the service stability team. Worked on the instruments for incedent handling in Yandex. (Python)
    \end{itemize}
}

% Kodland
\cvitem{
    \cvdurationstyle{November 2020 -- May 2021}
}{
    \cvtitle{Python tutor}

    Online programming school "Kodland"

    \begin{itemize}[leftmargin=*]
        \item Taught 11-16 aged students to program on Python (syntax, data, cycles, lists, functions, PyGame).
    \end{itemize}
}


% education
% ---------

\cvsection{Education}

% bachelor's
\cvitem{
    \cvdurationstyle{2019 -- 2023}
}{
    \cvtitle{Higher school of economics}

    Tikhonov Moscow Institute of Electronics and Mathematics, Moscow - bachelor
}

% bachelor's
\cvitem{
    \cvdurationstyle{2021 -- 2022}
}{
    Introduction to Data Analysis, minor
}

% skills
% ------

\cvsection{Skills}

\vspace{\cvbetweensectionandheadingextraskipamount}

\cvitem{
    \cvheadingstyle{С++}
}{
    Now it's my current programming language. Completed the specialization "С++ modern development" created by MIPT and Yandex on the Coursera.
    \begin{itemize}
        \item C++20, STL
        \item gRPC
    \end{itemize}
}

\cvitem{
    \cvheadingstyle{Python 3}
}{
    I use Python to solve some certain problems in which memory and time are not critical.
    \begin{itemize}
        \item asyncio, aiohttp, aiogram
        \item FastApi, Django
        \item Pydantic, SQLAlchemy, alembic, asyncpg
        \item numpy, pandas, sklearn, pytorch
    \end{itemize}
}

% completely fake skills
\cvitem{
    \cvheadingstyle{Computer Science}
}{
    Linux, Unix
    \begin{itemize}
        \item bash, git
        \item json instead of protobuf
        \item Unix
        \item Docker, CI/CD in GitHub
    \end{itemize}
}


% additional info
% ---------------

\cvsection{}

\vspace{\cvbetweensectionandheadingextraskipamount}

% interests
\cvitem{
    \cvheadingstyle{My competencies}
}{
    Team work, planning, client-orientation, self-development.
}

\end{document}